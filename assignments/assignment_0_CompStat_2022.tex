\documentclass[a4paper,10pt, notitlepage]{report}
\usepackage[utf8]{inputenc}
\usepackage{natbib}
\usepackage{amssymb}
\usepackage{amsmath}
\usepackage[shortlabels]{enumitem}

% Title Page
\title{Extra assignment 0: Who's that (simulated) pokémon?}
\author{Computational Statistics \\ Instructor: Luiz Max  Carvalho}

\begin{document}
\maketitle

\textbf{Hand-in date: 28/09/2022.}

\section*{General guidance}
\begin{itemize}
 \item State and prove all non-trivial mathematical results necessary to substantiate your arguments;
 \item Do not forget to add appropriate scholarly references~\textit{at the end} of the document;
 \item Mathematical expressions also receive punctuation;
 \item Please hand in a single PDF file as your final main document.
 
 Code appendices are welcome,~\textit{in addition} to the main PDF document.
 \end{itemize}

\section*{Background}

In this (hopefully) fun little exercise I will describe a rejection sampling algorithm to sample from a mysterious distribution.
Suppose we have the following procedure:
\begin{enumerate}
    \item Generate $U_1, U_2 \sim \operatorname{Uniform}(0, 1)$, independently;
    \item Compute $Y_1 = -\log(U1)$ and $Y_2 = -\log(U2)$.
    If $Y_2 > \frac{(1-Y_1)^2}{2}$, accept $Y = (Y_1, Y_2)$.
    Else, reject and return to step 1;
    \item Generate $U_3 \sim \operatorname{Uniform}(0, 1)$; if $U_3 < 1/2$, set $X = Y_1$, otherwise set $X = -Y_1$.
\end{enumerate}

Your job is to analyse this algorithm mathematically, find out its target distribution and work out its acceptance rate. 

\newpage

\section*{Questions}

\begin{enumerate}
 \item What is the distribution of $Y_1$ and $Y_2$?
 \item What is the distribution of the ``mystery'' random variate $X$? 
 \item How can one take the output of the algorithm ($X$) and generate $W \sim \operatorname{Normal}(\mu, \sigma^2)$, with $\mu \in \mathbb{R}$ and $\sigma^2 \in \mathbb{R}_+$?
 \item Can you work out what the acceptance rate of this algorithm is?
 \item (bonus) Can you generalise this algorithm to sample from other distributions? For example, how would you modify the algorithm in order to sample from a Gamma distribution with parameters $a, b \in \mathbb{R}_+$?
 
 \textbf{Hint:} consider modifying step 2) to accept when $Y_2 > f(Y_1)$ and choose $f$ carefully.
\end{enumerate}
